\documentclass{article}
\usepackage{amsmath}

\begin{document}

\title{Ground State Simulation of a Particle in a Box Using Power Iteration}
\author{Author's Name}
\date{\today}
\maketitle

\section{Introduction}
We present a simple C++ implementation for finding the ground state of
a quantum particle in a one-dimensional box using power iteration. The
box is discretized into $N$ points and the Hamiltonian of the system
is approximated as a sparse, tridiagonal matrix.

\section{Method}
The Hamiltonian of the particle in the box is given by
\begin{equation}
H = -\frac{{\hbar}^2}{2m} \frac{d^2}{dx^2},
\end{equation}
where $\hbar$ is the reduced Planck constant, $m$ is the mass of the
particle, and $x$ is the position. For simplicity, we have set $\hbar
= m = 1$.

We discretize the interval $0 < x < L$ into $N$ points and approximate
the second derivative using finite differences. The Hamiltonian is
then represented as a sparse matrix, $H$, of size $N \times N$.

To find the ground state of the system, we use the power iteration
method. We initialize a random vector, $\psi$, and update it by
repeated multiplication with $H$ and normalization.

\section{Implementation}
The code uses the Armadillo C++ library for linear algebra
operations. A sparse matrix is created for the Hamiltonian, reducing
memory requirements and improving the efficiency of the matrix-vector
multiplication in the power iteration method. After the power
iteration converges, the ground state energy and wavefunction of the
system are calculated.

\end{document}
